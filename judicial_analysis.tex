\documentclass[12pt]{article}
\usepackage[top=1in, bottom=2in, right=1in, left=1in]{geometry}
\usepackage{graphicx}
\usepackage{url}
\usepackage{amsmath,amssymb}
\usepackage{float}
\title{Judicial Independence as a Function of Political Competition: Comparing Democracies Over Time}
\author{David Carlson}
\date{Fall 2014}
\linespread{1.6}
\begin{document}
\maketitle

\section*{Argument}
In consolidated democracies, as political competition increases, the legislative branch becomes less productive. This lack of productivity creates a demand for judicial branches to be more independent and engage in policy making. When interest groups or citizens seek policy outcomes, if the legislature is consistently unable to provide the desired outcome, actors will turn to the judiciary. The judiciary, in an environment in which outdated legislation is being applied to an evolving reality, will reinterpret legislation, effectively creating policy in the process. This exercise of independence is less constrained by the other branches, because the other branches cannot overcome the collective action problems necessary to overrule the judiciary, and there exists no clear and unifying preferred policy position within the other branches on which to gauge the policy created by the courts. Therefore, judicial independence, arguably, is not a result of conscious decision-making by the political leaders, but instead a demand-driven process and the result of a relative increase in \textit{de facto} power as compared to the power of the other branches, driven by both the demand from citizens for policy change, and the judiciary seeking greater influence.

Extant literature is divided on this topic and there has not been an adequate empirical, large-N test of the proposed relationship. Furthermore, various causal mechanisms have been proposed that could lead to such a relationship, which also have not been adequately tested. Specifically, the arguments of Hirschl (2000) and Ginsburg (2003), focused on the transition to a democracy, involve the institutionalization of an independent judiciary by the pre-transition elites to maintain their hegemony in the political and economic environment. These arguments could extend to explain the variation in independence within consolidated democratic states over time by suggesting that parties in power may institutionalize independent judiciaries in anticipation of losing power. Similarly, Hanssen (2004) argues that political leaders create the independent judiciary to ensure stability in the legislation they passed in their tenure, but is silent as to why the successors wouldn't simply overturn the independence or engage in court-curbing measures. Whittington's (2003) argument addresses this, showing that court-curbing is more common in the U.S. Congress when the party is more dominant. Is the successful creation and maintenance of an independent judiciary then reliant not only on the incumbent anticipating losing power, but on the power turnover to not be so drastic as to shift from one party or coalition extreme domination to another? Not only would competition be driving judicial independence, but there would need to be extreme domination by a political party, enough to enact preferred policies and establish an independent judiciary, followed by a period of power sharing so as to not allow one ruling party to curb the court. In this conception, then, judicial independence would be the result of a transition from one-party domination to a more competitive system, not at all unlike the transition from an authoritarian regime to a democracy. It is possible these unique circumstances could occur and could influence the level of judicial independence observed in a state. The peculiarities and specificity of this scenario, however, undermine the likelihood that this is the driving force behind the variation seen in judicial independence across modern democratic states. Nevertheless, to explore this alternative explanation, as will be further explained below, the tests will not only test the relationship between competition and independence, but also between volatility and independence.



\section*{Data}
Because this paper is only proposing an argument related to democratic countries, it is necessary to separate democratic from undemocratic states. To do this, Polity IV data will be used. Any state achieving a score of 6 or higher in the polity rating is considered a democracy and will be included in this analysis. Additionally, to ensure the effects of regime transitions are not captured by the analysis, only democracies that have survived for at least five years will be included.

\subsection*{Outcome Variable}
The outcome of interest is the level of \textit{de facto} judicial independence. The theoretical argument rests on the evolution of a system creating the actual practice of judicial independence, not on the changes in legislation that support an independent judiciary. Arguments more in line with Hanssen's reasoning could be focused on measuring the degree of \textit{de jure} independence, which might be of interest to understanding legislative behavior, but it is outside the scope of this paper. Secondly, a measure that is taken over time is paramount. If a correlation is observed, the driving logic behind the correlation is not necessarily discovered. Therefore, time series data across countries is fundamental to the understanding of the problem. For example, if increases in political competition do not precede increases in judicial power, the argument that the anticipation of losing power is the driving force behind the correlation loses much weight. In a system where a leader decides to increase judicial independence as a means of maintaining her own policy preferences, then the measures for competition, incorporating turnover and party vote shares, would increase after the establishment of an independent judiciary. Comparing lagged and non-lagged regressions will be quite valuable to gaining a better understanding of the underlying story.

There is no one-size-fits-all dataset measuring judicial independence. Some measures attempt to measure \textit{de jure} independence, while others attempt to measure \textit{de facto} judicial independence. Some measures are continuous while others are binary or categorical. These data can come from surveys, state reports, or are considered a proxy rather than a direct measure. Different data are available for different countries and different time periods. It is important for scholars to be aware of the strengths and weaknesses of the various data and the specific data generating processes in order to choose, and defend, an appropriate measure. This is of particular importance given the inherent intangibility and difficulty of measuring an idea without a clear and universal definition. The data created by Linzer and Staton (2011) combines multiple measures into one continuous proportion measurement of \textit{de facto} independence, with 0 being not independent and 1 being fully independent\footnote{For a more detailed account of the data generating process, see Linzer, Drew and Jeffrey Staton. 2011. ``A Measurement Model of Synthesizing Multiple Comparative Indicators: The Case of Judicial Independence.'' Working Paper, Emory University.}. They model the measurement, allowing for estimation and a measure of uncertainty. The uncertainty measure is the 80\% confidence interval of the estimation. Because this variable is the outcome variable, this uncertainty need not be accounted for as the error will capture this (see Treier and Jackman 2008). They provide data on 200 countries from 1960 to 2010. This is a very flexible measurement ideal for testing a plethora of theories. The inclusion of many measures theoretically includes much more information, minimizing uncertainty. This is the most appropriate data to test the theory of this paper.

\subsection*{Independent Variable}
Three different measures will be to used to assess the degree of political competition within a country. This helps alleviate some of the potential measurement errors by testing the theory using data generated through different processes. It also allows to test the theory on a wider range of countries and over a longer period of time, because data for each measure is not available for every country-year provided in the Linzer-Staton database. The first comes from the World Bank Database of Political Institutions 2012 (Thorsten et al)\footnote{Available at: \url{http://www.nsd.uib.no/macrodataguide/set.html?id=11&sub=1}}. The first measure, frac, is a measure of fractionalization within a legislature. More specifically, it is the probability that two deputies picked at random from the legislature will be of different parties. This measure is appropriate, because according to the theory of this paper, when there is greater fractionalization there is greater competition in the sense that one party cannot control the agenda. The data subset including this measure includes 872 observations over 44 countries.

The next variable, contest, to be used is a principal component factor index of many political contestation measures. Teorell et al (2008) developed this measure as an indicator of real political competition within a country. Higher values indicate more frequent turnovers through elections and greater power sharing between parties.\footnote{Data obtained from Quality of Government Institute Standard Data: \url{http://www.qog.pol.gu.se}}. The subset of data including this measure includes 574 observations over 37 countries. Finally, a variable to indicate volatility, vol, will be used to better flesh out the causal mechanisms\footnote{Data available at: \url{http://www.eleanorneffpowell.com/data.html}}. This measure of volatility measures the degree to which an election changed the party composition of a legislature. It is highest when a drastic power shift occurs, more specifically when one party loses a very large majority at an election and is replaced by another party gaining a very large majority. It is smallest when no changes occur in party composition within the legislature. Unfortunately, this data is only available for European countries on election years, so it will not be useful as an adequate test. The subset of data including this measure inludes 55 observations of 14 countries, with between 4 and 6 observations per country typically. However, it is suggestive. If volatility is positively correlated with judicial independence, Hanssen's (2004) theory and some of the other earlier mentioned theories might gain some weight. If parties are creating independent judiciaries in order to maintain policy preferences or to create stability in a power sharing arrangement, then high levels of volatility will be associated with judicial independence. However, if instead political competition is linked to judicial independence in the way proposed by this paper, high levels of volatility should not be associated with judicial independence. High levels indicate that an election caused one party to lose power substantially and another to gain power substantially. This would imply that a party would have sufficient power to curb the judiciary.

Because not all of these measures are available for each country-year in the subsetted Linzer and Staton data, it is appropriate to use multiple imputation to alleviate some of the bias introduced by non-random missingness. Because the volatility index is missing a very high proportion of data, imputation on this variable could be unwise. However, the results shown below were repeated with case-wise deletion for this variable, and the results are almost identical.

\begin{table}[ht]
\centering
\begin{tabular}{rrrr}
  \hline
 & frac & contest & vol \\ 
  \hline
1 &  62 & 334 & 877 \\ 
   \hline
\end{tabular}
\caption{Missingness} 
\end{table}


\subsection*{Controls}
The first control to be used is the Polity IV rating of democratic governance. It is conceivable that high levels of democracy cause both increased measures of judicial independence and increased degree of political competition. Though all states considered are democracies, the level of democracy varies. The second control will be the length of democratic governance\footnote{Both these measures available at \url{http://www.systemicpeace.org/inscrdata.html}}. Again, it is conceivable that the length of democratic governance directly affects the level of judicial independence and the degree of political competition. A measurement for presidential systems versus parliamentary systems will  also be used. The variable takes a 2 if the system is parliamentary, 1 if there is an assembly-elected president, and 0 if it is a presidential system. This variable was treated as continuous as there seems to be a logical ordering of the values\footnote{Regressions were run treating this as categorical with no substantive changes.}. A presidential system may impact the amount of political competition by unifying strong parties, and it may impact the amount of judicial independence by creating an additional check on the judiciary\footnote{A dummy was considered for PR systems, but the effect of this variable was not statistically reliable, the magnitude of the estimate was very close to zero, and inclusion did not statistically improve model fit.}. Finally, a dummy indicating type of colonial history will be used. Colonial rule undoubtedly has an impact on political institutions, and it is possible that a particular colonial history could impact both political competition and judicial independence\footnote{These data available at \url{http://siteresources.worldbank.org/INTRES/Resources/469232-1107449512766/DPI2012.dta}}.


\section*{Data Analysis}
The data to be analyzed is time-series data across countries. Therefore, a multilevel model will be used with varying intercepts for each country. The effects of the controls and the independent variables of interest will be fixed. Each model uses a different independent variable of interest. 


\begin{table}[ht]
\centering
\begin{tabular}{rrrrrrr}
  \hline
 & mean & se & t value & Pr($>$$|$t$|$) & 2.5\% & 97.5\% \\ 
  \hline
system & 0.11 & 0.06 & 1.94 & 0.03 & -0.00 & 0.22 \\ 
  polity & 0.17 & 0.02 & 9.34 & 0.00 & 0.13 & 0.20 \\ 
  years & 0.01 & 0.00 & 8.22 & 0.00 & 0.01 & 0.01 \\ 
  frac & 0.32 & 0.12 & 2.54 & 0.01 & 0.07 & 0.56 \\ 
  deviance & 263.74 & 10.65 & 24.77 & 0.00 & 242.87 & 284.61 \\ 
  g.0 & -0.30 & 0.30 & -1.00 & 0.16 & -0.90 & 0.29 \\ 
  Colonial\_Spain & -1.42 & 0.36 & -3.98 & 0.00 & -2.12 & -0.72 \\ 
  Colonial\_UK & -1.15 & 0.37 & -3.07 & 0.00 & -1.89 & -0.42 \\ 
  Colonial\_Portugal & -1.59 & 0.58 & -2.74 & 0.00 & -2.73 & -0.45 \\ 
  Colonial\_France & -1.49 & 0.95 & -1.56 & 0.06 & -3.35 & 0.38 \\ 
  sigma.a & 0.92 & 0.11 & 8.28 & 0.00 & 0.70 & 1.13 \\ 
  sigma.y & 0.28 & 0.01 & 41.96 & 0.00 & 0.27 & 0.29 \\ 
   \hline
Avg. DIC:&316.747\\
\hline
\end{tabular}
\caption{Model with frac} 
\end{table}

\begin{table}[ht]
\centering
\begin{tabular}{rrrrrrr}
  \hline
 & mean & se & t value & Pr($>$$|$t$|$) & 2.5\% & 97.5\% \\ 
  \hline
system & 0.11 & 0.06 & 1.93 & 0.03 & -0.00 & 0.22 \\ 
  polity & 0.15 & 0.02 & 7.71 & 0.00 & 0.11 & 0.19 \\ 
  years & 0.01 & 0.00 & 9.22 & 0.00 & 0.01 & 0.01 \\ 
  contest & 0.27 & 0.10 & 2.64 & 0.01 & 0.07 & 0.47 \\ 
  deviance & 258.57 & 12.68 & 20.39 & 0.00 & 233.71 & 283.42 \\ 
  g.0 & -0.37 & 0.31 & -1.21 & 0.11 & -0.97 & 0.23 \\ 
  Colonial\_Spain & -1.35 & 0.35 & -3.89 & 0.00 & -2.03 & -0.67 \\ 
  Colonial\_UK & -1.09 & 0.36 & -3.01 & 0.00 & -1.81 & -0.38 \\ 
  Colonial\_Portugal & -1.52 & 0.56 & -2.70 & 0.00 & -2.63 & -0.42 \\ 
  Colonial\_France & -1.38 & 0.92 & -1.50 & 0.07 & -3.18 & 0.42 \\ 
  sigma.a & 0.88 & 0.11 & 8.15 & 0.00 & 0.67 & 1.10 \\ 
  sigma.y & 0.28 & 0.01 & 41.53 & 0.00 & 0.26 & 0.29 \\ 
   \hline
Avg. DIC:&311.272\\
\hline
\end{tabular}
\caption{Model with contest} 
\end{table}

\begin{table}[ht]
\centering
\begin{tabular}{rrrrrrr}
  \hline
 & mean & se & t value & Pr($>$$|$t$|$) & 2.5\% & 97.5\% \\ 
  \hline
system & 0.09 & 0.06 & 1.54 & 0.06 & -0.03 & 0.21 \\ 
  polity & 0.15 & 0.02 & 6.66 & 0.00 & 0.11 & 0.20 \\ 
  years & 0.01 & 0.00 & 7.80 & 0.00 & 0.01 & 0.01 \\ 
  vol & -0.00 & 0.00 & -0.71 & 0.25 & -0.01 & 0.00 \\ 
  deviance & 254.40 & 39.53 & 6.44 & 0.00 & 176.92 & 331.87 \\ 
  g.0 & 0.13 & 0.40 & 0.32 & 0.38 & -0.66 & 0.92 \\ 
  Colonial\_Spain & -1.39 & 0.35 & -3.92 & 0.00 & -2.08 & -0.69 \\ 
  Colonial\_UK & -1.16 & 0.37 & -3.13 & 0.00 & -1.88 & -0.43 \\ 
  Colonial\_Portugal & -1.61 & 0.57 & -2.82 & 0.00 & -2.73 & -0.49 \\ 
  Colonial\_France & -1.38 & 0.93 & -1.48 & 0.07 & -3.21 & 0.44 \\ 
  sigma.a & 0.90 & 0.11 & 8.08 & 0.00 & 0.68 & 1.11 \\ 
  sigma.y & 0.28 & 0.01 & 32.26 & 0.00 & 0.26 & 0.29 \\ 
   \hline
Avg. DIC:&307.331\\
\hline
\end{tabular}
\caption{Model with vol} 
\end{table}

\begin{table}[ht]
\centering
\begin{tabular}{rrrrrrr}
  \hline
 & mean & se & t value & Pr($>$$|$t$|$) & 2.5\% & 97.5\% \\ 
  \hline
system & 0.10 & 0.06 & 1.84 & 0.04 & -0.01 & 0.22 \\ 
  polity & 0.17 & 0.02 & 9.28 & 0.00 & 0.13 & 0.20 \\ 
  years & 0.01 & 0.00 & 9.15 & 0.00 & 0.01 & 0.01 \\ 
  deviance & 271.55 & 10.15 & 26.75 & 0.00 & 251.65 & 291.44 \\ 
  g.0 & -0.01 & 0.28 & -0.03 & 0.49 & -0.55 & 0.53 \\ 
  Colonial\_Spain & -1.41 & 0.35 & -3.96 & 0.00 & -2.10 & -0.71 \\ 
  Colonial\_UK & -1.16 & 0.37 & -3.12 & 0.00 & -1.89 & -0.43 \\ 
  Colonial\_Portugal & -1.62 & 0.58 & -2.79 & 0.00 & -2.75 & -0.48 \\ 
  Colonial\_France & -1.40 & 0.94 & -1.49 & 0.07 & -3.24 & 0.44 \\ 
  sigma.a & 0.91 & 0.11 & 8.30 & 0.00 & 0.69 & 1.12 \\ 
  sigma.y & 0.28 & 0.01 & 41.98 & 0.00 & 0.27 & 0.29 \\ 
   \hline
DIC:&323.078\\
\hline
\end{tabular}
\caption{Null Model} 
\end{table}

\section*{References}
\noindent Bates, Douglas, Martin Maechler, Ben Bolker, Steven Walker, Rune Haubo Bojesen Christensen, and Henrik Singmann. 2014. lme4: Linear mixed-effects models using Eigen and S4. R package version 1.1-6.\\

\\\noindent Callander, Steven and Krehbiel, Keith, Gridlock and Delegation in a Changing World. 2014. Forthcoming in the American Journal of Political Science; Stanford School of Business Research Paper No. 2100.\\

\\\noindent Cameron, Charles. 2002. ``Judicial Independence: How Can You Tell It When You See It? And, Who Cares?'' In S Burbank and B Friedman (eds). \textit{Judicial Independence
at the Crossroads.} Sage Publications. pp. 134-147.\\

\\\noindent Chavez, Rebecca Bill 2003. ``The Construction of the Rule of Law in Argentina: A Tale of Two Provinces.'' Comparative Politics, Vol. 35, No. 4 (Jul., 2003), pp. 417-437.\\

\\\noindent Coppedge, Alvarez, and Maldonado. 2008. ``Two Persistent Dimensions of Democracy: Contestation and Inclusiveness.'' Under Review.

\\\noindent Epstein, Lee, Jack Knight, and Olga Shvetsova. 2001. ``The Role of Constitutional Courts in the Establishment and Maintenance of Democratic Systems of Government.'' Law and Society Review 35:117-163.\\

\\\noindent Figueroa, J. Rios. 2007. ``Fragmentation of Power and the Emergence of an Effective Judiciary in Mexico, 1994-2002.'' Latin American Politics and Society 49: 31-57.\\

\\\noindent Ginsburg, Tom. 2003. \textit{Judicial Review in New Democracies.} New York: Cambridge University Press.\\

\\\noindent Graber, Mark A. 1993. ``The Nonmajoritarian Difficulty: Legislative Deference to the Judiciary.'' Studies in American Political Development, 7, pp 35-73.\\

\\\noindent Hlavac, Marek (2014). stargazer: LaTeX code and ASCII text for well-formatted regression and summary statistics tables. R package version 5.0. \url{http://CRAN.R-project.org/package=stargazer}\\

\\\noindent Hanssen, F. Andrew. 2004. ``Is There a Politically Optimal Level of Judicial
Independence?'' American Economic Review 94(3): 712-729.\\

\\\noindent Helmke, Gretchen. 2005. \textit{Courts Under Constraints.} Cambridge University Press.\\

\\\noindent Hirschl, Ran. 2000. ``The Political Origins of Judicial Empowerment Through
Constitutionalization: Lessons from Four Constitutional Revolutions.'' Law and
Social Inquiry 25:91-147.\\

\\\noindent Kornhauser, Lewis. 1992. ``Modeling Collegial Courts: Legal Doctrine.'' Journal of Law, Economics, and Organization 8: 441-70.\\

\\\noindent Landes, William and Richard Posner. 1975. ``The Independent Judiciary in an Interest-Group Perspective.'' Journal of Law and Economics 18(3): 875-901.\\

\\\noindent Linzer, Drew and Jeffrey Staton. 2011. ``A Measurement Model of Synthesizing Multiple Comparative Indicators: The Case of Judicial Independence.'' Working Paper,
Emory University.\\

\\\noindent Powell, Eleanor Neff and Joshua A. Tucker. 2013. ``Revisiting Electoral Volatility in Post-Communist Countries: New Data, New Results and New Approaches." British Journal of Political Science, Available on CJO 2013 doi:10.1017/S0007123412000531.\\

\\\noindent Ramseyer, J. Mark. 1994. ``The Puzzling (In)dependence of the Courts: A Comparative Approach.'' Journal of Legal Studies 23: 721.\\

\\\noindent Rodriguez-Raga, Juan Carlos. 2011. ``Strategic Deference in the Colombian Constitutional Court.'' In G Helmke and JR Figueroa (eds). \textit{Courts in Latin America.} Cambridge University Press.\\

\\\noindent Stephenson, Matthew C. 2003. ``When the Devil Turns: The Political Foundations of Independent Judicial Review.'' 32 Journal of Legal Studies 59, 62-63.\\

\\\noindent Teorell, Jan, Nicholas Charron, Stefan Dahlberg, Sören Holmberg, Bo Rothstein, Petrus Sundin & Richard Svensson. 2013. The Quality of Government Dataset, version 20Dec13. University of Gothenburg: The Quality of Government Institute, \url{http://www.qog.pol.gu.se}\\

\\\noindent Thorsten Beck, George Clarke, Alberto Groff, Philip Keefer, and Patrick Walsh, 2001. ``New tools in comparative political economy: The Database of Political Institutions.'' 15:1, 165-176 (September), World Bank Economic Review.\\

\\\noindent Treier, Shawn and Simon Jackman. 2008 ``Democracy as a Latent Variable.'' \textit{American Journal of Political Science}. 52(1): 201-217.\\

\\\noindent Trochev, Alex. 2004. ``Less Democracy, More Courts: A Puzzle of Judicial Review in Russia.'' Law and Society Review 38(3): 523-548.\\

\\\noindent Vanberg, Georg. 2005. \textit{The Politics of Constitutional Review in Germany.} Cambridge University Press.\\

\\\noindent Vanberg, Georg. 2008. ``Establishing and Maintaining Judicial Independence.'' In KE Whittington, RD Kelemen and GA Caldeira (eds.) \textit{The Oxford Handbook of Law and Politics.} Oxford University Press.\\

\\\noindent Whittington, Keith E. 2003. ``Legislative Sanctions and the Strategic Environment of Judicial Review.'' International Journal of Constitutional Law 1 (July): 446���74.\\

\\\noindent Whittington, Keith. 2005. ``Interpose Your Friendly Hand: Political Supports for the Exercise of Judicial Review by the United States Supreme Court'' American Political Science Review 99(4): 583-596.\\




\end{document}
